\documentclass[12pt,a4paper]{article}
\usepackage[utf8]{inputenc}
\usepackage[german]{babel}
\usepackage[T1]{fontenc}
\usepackage{amsmath}
\usepackage{amsfonts}
\usepackage{amssymb}
\usepackage{listings}
\usepackage[onehalfspacing]{setspace}
\lstset{
	language=C,
	numbers=left,                   % where to put the line-numbers
	stepnumber=1,                   % the step between two line-numbers.        
	numbersep=5pt,                  % how far the line-numbers are from the code
	backgroundcolor=\color{white},  % choose the background color. You must add \usepackage{color}
	showspaces=false,               % show spaces adding particular underscores
	showstringspaces=false,         % underline spaces within strings
	showtabs=false,                 % show tabs within strings adding particular underscores
	tabsize=2,                      % sets default tabsize to 2 spaces
	captionpos=b,                   % sets the caption-position to bottom
	breaklines=true,                % sets automatic line breaking
	breakatwhitespace=true,         % sets if automatic breaks should only happen at whitespace
	title=\lstname,    
	commentstyle=\color{green}, % comment color
	keywordstyle=\color{blue}, % keyword color
	stringstyle=\color{red} % string color 
}
\usepackage[colorlinks,
pdfpagelabels,
pdfstartview = FitH,
bookmarksopen = true,
bookmarksnumbered = true,
linkcolor = black,
plainpages = false,
hypertexnames = false,
citecolor = black] {hyperref}
\author{Henrik Erhart, Tobias Drewelies}
\title{Valgrind in der Qualitätssicherung}
\begin{document}
\maketitle
\tableofcontents
\newpage
\section{Einführung}
Bei der Erstellung komplexerer C und C++ Programme entstehen meist unweigerliche menschliche Fehler und Bugs. Es ist leicht passiert, dass der Programmierer vergisst, ein dynamisch allokierten Speicherbereich frei zu geben oder bei einem mutlithread Programm ein ungewollter Deadlock entsteht. Diese Fehler sind sehr kritisch und führen meist zum Absturz des Programms. Das manuelle debuggen dieser Art der Fehler stellt sich oft schwierig dar und ist nicht immer mit dem bloßem Auge zu erkennen. Was also nun? Hierfür soll das Framework Valgrind Abhilfe schaffen. Ursprünglich als Memory Checker, 2002 veröffentlich, ist Valgrind heute ein Meta-Tool zur Erstellung von Überwachungsprogrammen für C und C++ Programme.

\subsection{Anwendung in der Qualitätssicherung}
Da Speicherlecks, Race-Conditions und ähnliches keine Warnings im Compiler erzeugen und nur sehr selten zu einem Fehler führen, ist es wichtig diese automatisch zu erkennen. 

Mit den Werkzeugen der Valgrind-Suite werden solche Fehler automatisch gefunden. Dies erspart Zeit, die sonst durch manuelle Prüfung und Kontrollen verloren ginge. Auch können so schnell die Ursachen für Fehler gefunden werden, die zum Beispiel nur in einem von Tausend Durchläufen oder noch seltener auftreten.

Da Valgrind nur über die Konsole gesteuert wird, lassen sich besonders einfach Scripte verwenden mit denen ein automatischer Testablauf auf einem Testserver gestaltet werden kann. So eingesetzt, kann mit Hilfe der Werkzeuge der Valgrind-Suite stets sichergestellt werden, dass Neuerungen im Programm zu keinen Speicherlecks oder Race-Conditions geführt haben. 

Auch die Optimierung von Code kann durch Werkzeuge wie Cachegrind erleichtert werden, da hier besonders ineffiziente Speicherzugriffe erkannt werden können. Dies ist insbesondere bei der Entwicklung für eingebettete Systeme wichtig, die hier oft nicht genügend Rechenleistung zur Verfügung steht um teure Cache-Misses zu kaschieren. 

\newpage
\section{Der Aufbau des Frameworks}
Valgrind besteht zum einen aus dem Valgrind core, welcher die essentiellen Bestandteile zum überwachten Ausführen von Programmen bereitstellt und zum anderen aus verschiedenen sogenannten Skins, welche mit dem core über ein Interface interagieren und einer speziellen Aufgabe dienen, wie beispielsweise dem überprüfen des Speichers. Diese Skins können den speziellen Anwendungsfällen angepasst und erweitert werden. Die etablierten Standard Skins die öffentlich zur Verfügung stehen umfassen die so genannten Tools Memecheck, Addcheck, Chachegrind, Helgrind und Nulgrind.

\subsection{Der Core}
Der Core von Valgrind liest executables eines zu untersuchenden Programms ein und führt diese auf einer virtuellen x86 CPU aus. Ein JIT Compiler annotiert dann jede Anweisung mit weiteren Befehlen und führt dabei jede einzelne Anweisung aus und simuliert sie. Für die meisten Skins ist es dabei wichtig, dass beim kompilieren debugg Informationen mitgegeben werden, da diese von den Valgrind Tools ausgelesen und verwendet werden. Dadurch wird die dynamische Analyse des Programmverhaltens ermöglicht, im Gegensatz zu statischen Analysen, welche meist lediglich anhand des Quelltext versuchen Fehler zu ermitteln. (Quelle: Heise.de)

Durch dieses Verfahren ergibt sich der Vorteil, dass der gesamte Code des Programms samt aller benutzten Bibliotheken überwacht werden kann, ohne den Quelltext zu benötigen.
Ebenso müssen auch die zu überwachenden Programme nicht neu Kompiliert oder neu Gelinkt werden.\cite{valarticle}

Obwhol Valgrind hauptsächlich zur Überwachung von C/C++ Programmen eingesetzt wird, findet es auch in anderen Bereichen Anwendung. In Java wird die Speicherverwaltung zwar schon durch den integrierten Grabadge Collector vereinfacht, doch dennoch kann Valgrind auch Java Progrmmieren helfen Fehler, wie sie beispielsweise bei der Multithread programmieren entstehen können, ausfindig zu machen. Dazu ist Valgrind inder der Lage, da es so konzipiert ist, dass es kompilierten Programmcode aus beliebigen Sprachen verarbeiten kann.
\newpage
\section{Memcheck}
Memcheck ist das erste und wohl bekannteste Tool, das mit dem Valgrind-Framework ausgeliefert wird. Oft meinen Nutzer wenn sie von Valgrind reden eigentlich dieses Werkzeug. Es dient dazu in mit C und C++ geschriebenen Programmen sogenannte Speicherlecks oder sonstige fehlerhafte Speicherzugriffe aufzudecken. Auch der Zugriff auf noch nicht initialisierte Variablen kann mit diesem Werkzeug erkannt werden.

In großen Programmen kann es schnell vorkommen, dass einmal reservierte Speicherbereiche nicht mehr, oder nur in seltenen Ausnahmefällen nicht mehr frei gegeben werden. Diese Fehler können mit herkömmlichen Debuggern nur schwer bis gar nicht gefunden werden. Oft sind die Speicherlecks auch so klein, dass sie lediglich wenige Bytes pro Minute oder Stunde groß sind. Dies kann in vielen Anwendungen kein Problem sein, bei solchen allerdings, die Tagelang laufen, kann es sein, dass solche Lecks dafür sorgen, dass der benötigte Speicher den zur Verfügung stehenden Speicher überschreitet und es zu Performance-Verlust oder gar Fehlern kommt. Memcheck kann in solchen fällen helfen die Ursachen für\onehalfspacing solche Lecks zu finden und zu beheben.

\subsection{Beispiels}
Das folgende Beispiel zeigt einige typische Fehler im Umgang mit Speicher und wie Memcheck sie ausgibt
\begin{singlespace}
\begin{scriptsize}
\lstinputlisting{../Examples/memcheck/memchek_example.c}

\begin{lstlisting}
	==8769== Memcheck, a memory error detector
	==8769== Copyright (C) 2002-2015, and GNU GPL d, by Julian Seward et al.
	==8769== Using Valgrind-3.11.0 and LibVEX; rerun with -h for copyright info
	==8769== Command: ./memchek_example
	==8769== 
	==8769== Invalid read of size 4
	==8769==    at 0x400559: foo (memchek_example.c:12)
	==8769==    by 0x40058E: main (memchek_example.c:20)
	==8769==  Address 0x520403c is 4 bytes before a block of size 40 alloc d
	==8769==    at 0x4C2DB8F: malloc (in /usr/lib/valgrind/vgpreload_memcheck-amd64-linux.so)
	==8769==    by 0x400537: foo (memchek_example.c:8)
	==8769==    by 0x40058E: main (memchek_example.c:20)
	==8769== 
	==8769== Invalid write of size 4
	==8769==    at 0x400577: foo (memchek_example.c:13)
	==8769==    by 0x40058E: main (memchek_example.c:20)
	==8769==  Address 0x5204068 is 0 bytes after a block of size 40 alloc d
	==8769==    at 0x4C2DB8F: malloc (in /usr/lib/valgrind/vgpreload_memcheck-amd64-linux.so)
	==8769==    by 0x400537: foo (memchek_example.c:8)
	==8769==    by 0x40058E: main (memchek_example.c:20)
	==8769== 
	==8769== 
	==8769== HEAP SUMMARY:
	==8769==     in use at exit: 40 bytes in 1 blocks
	==8769==   total heap usage: 1 allocs, 0 frees, 40 bytes allocated
	==8769== 
	==8769== 40 bytes in 1 blocks are definitely lost in loss record 1 of 1
	==8769==    at 0x4C2DB8F: malloc (in /usr/lib/valgrind/vgpreload_memcheck-amd64-linux.so)
	==8769==    by 0x400537: foo (memchek_example.c:8)
	==8769==    by 0x40058E: main (memchek_example.c:20)
	==8769== 
	==8769== LEAK SUMMARY:
	==8769==    definitely lost: 40 bytes in 1 blocks
	==8769==    indirectly lost: 0 bytes in 0 blocks
	==8769==      possibly lost: 0 bytes in 0 blocks
	==8769==    still reachable: 0 bytes in 0 blocks
	==8769==         suppressed: 0 bytes in 0 blocks
	==8769== 
	==8769== For counts of detected and suppressed errors, rerun with: -v
	==8769== ERROR SUMMARY: 3 errors from 3 contexts (suppressed: 0 from 0)
	
\end{lstlisting}
\end{scriptsize}
\end{singlespace}
\newpage
In der Ausgabe des Programms ist zu erkennen, dass Memcheck in dem Programm, dass zuvor weder vom Compiler bemängelt wurde, noch Fehler während der Ausführung hatte, die fehlerhafte Speichernutzung erkennt.
Zunächst wird in jeder Zeile Links die Prozess-ID angegeben. Die ersten Ausgaben liefern verschiedenen Meta-Daten, wie die Version des Valgrind cores, die Lizenz unter der das Tool veröffentlicht ist und das verwendete Tool, in diesem Fall Memcheck. Darunter wird dann noch das getestete Programm ausgegeben bevor die erkannten Warnungen und Fehlermeldungen ausgegeben werden. 

Aus den Warnungen und Fehlermeldung geht hervor, dass Memcheck alle von uns eingebauten Fehler erkannt hat. Zunächst wird ein \glqq Invalid read of size 4\grqq  vermeldet. Zudem liefert der darauffolgende Stack Trace Rückschlüsse darauf, an welche Stelle im Quelltext der Fehler verursacht wurde. In unserem Beispiel wird korrekt die Zeile angegeben in der wir auf einen nicht allokierten Speicherbereich zugreifen wollen. Ebenso wird das schreiben in einen nicht allokierten Speicherbereich erkannt. Zum Schluss in der "Heap Summary" sehen wir auch, das wir vergessen haben einen 40 Bytes großen Speicherbereich freizugeben durch die Ausgabe "40 bytes in 1 blocks are definitely lost in loss record 1 of 1". Diese wir nochmals in der darauffolgenden "Leak Summary" für uns zusammengefasst.


\newpage
\section{Cachegrind}
Cachegrind ist ein Werkzeug, um zu ermitteln, wie effizient Speicherzugriffe erfolgen. Dabei simuliert das Werkzeug, egal wie viele Cache-Ebenen es gibt, lediglich zwei Cache-Ebenen. Die erste und die letzte Cache-Ebene. In der ersten Ebene werden dabei ein getrennter Instruktions- und Datencache simuliert, während die letzte Cache-Ebene ein vereinheitlichter Cache ist.

Die erste Cache-Ebene wird simuliert, um Stellen aufzudecken an denen der Code schlecht mit der Architektur des Caches interagiert aufzudecken. Dies kann zum Beispiel sein, wenn auf eine Matrix, die Zeilenweise im Speicher liegt, spaltenweise zugegriffen wird. Der letzte Level des Caches wird deshalb abgebildet, weil er den Zugriff auf den Hauptspeicher puffert und bei einem Miss direkt aus diesem gelesen werden muss. Misses sind hier daher besonders teuer. 

Cachegrind bietet zusätzlich noch die Möglichkeit, Fehler bei Sprungvorhersagen zu ermitteln. Dazu muss das Programm mit dem Parameter --branch-sim=yes gestartet werden. Oft kann eine häufig falsche Sprungvorhersage durch einfache Umstrukturierungen des Quelltextes behoben werden, was die Performance des Programms verbessert. Dies ist insbesondere für Programme auf eingebetteten Geräten eine nützliche Optimierung, da die Sprungvorhersage auf kleinen Prozessoren im Gegensatz zu der auf Desktop-Prozessoren nur sehr einfach gehalten ist.

\subsection{Beispiel}
Im folgenden Beispiel werden zwei Programme gezeigt. Beide Programme legen eine 1000 mal 1000 große Matrix von ganzzahligen Werten an und initialisieren bis zu einem Zehntel der Werte. Das erste Programm initialisiert hierbei linear das erste Zehntel des allozierten Speichers mit Zufallszahlen. Das Zweite Programm initialisiert die Felder mit dem Index der Zufallszahl mit selbiger.

Lässt man Cachegrind die beiden Programme analysieren, so ist festzustellen, dass das Programm mit den zufälligen Zugriffen deutlich mehr Cache-Misses hervorbringt, als das mit dem linearen Zugriffen.
 
\begin{singlespace}
\begin{scriptsize}
\begin{lstlisting}
==5722== I   refs:      10,385,454
==5722== I1  misses:           780
==5722== LLi misses:           767
==5722== I1  miss rate:       0.00%
==5722== LLi miss rate:       0.00%
==5722== 
==5722== D   refs:       6,181,862  (4,065,234 rd   + 2,116,628 wr)
==5722== D1  misses:       101,300  (    1,930 rd   +    99,370 wr)
==5722== LLd misses:        60,700  (    1,611 rd   +    59,089 wr)
==5722== D1  miss rate:        1.6% (      0.0%     +       4.6%  )
==5722== LLd miss rate:        0.9% (      0.0%     +       2.7%  )
==5722== 
==5722== LL refs:          102,080  (    2,710 rd   +    99,370 wr)
==5722== LL misses:         61,467  (    2,378 rd   +    59,089 wr)
==5722== LL miss rate:         0.3% (      0.0%     +       2.7%  )
\end{lstlisting}
\end{scriptsize}
\end{singlespace}

In der obigen Ausgabe ist des Programms Cachegrind sind die Cache-Statistiken für das Programm mit den zufälligen Speicherzugriffen dargestellt. In den Zeilen 1 bis 5 sind die Zugriffe auf den Befehlscache angezeigt. Da das Beispielprogramm einen linearen Programmablauf hat, sind hier die Misses wie zu erwarten vernachlässigbar gering. In Zeile 1 wird dabei angezeigt, wie viele Instruktionen insgesamt ausgeführt wurden. In der zweiten Zeile sind die Misses im Level1 Cache aufgeführt, in Zeile 3 die im letzten Cache. Anschließend wird die prozentuale Fehlerrate für bei Caches angegeben. Ist hier eine größere Prozentzahl angegeben, so ist dies ein Indiz dafür, dass im Verlauf des Programms viele unerwartete Sprünge stattfinden, die es zu vermeiden gilt.

In den Zeilen 7 bis 11 sind die Statistiken für den Zugriff auf Daten angegeben. Diese sind analog zu den bereits erläuterten Statistiken des Befehls-Caches aufgebaut. Da Nutzdaten allerdings auch geschrieben werden können ist nach der Gesamtstatistik jeweils noch eine Aufteilung auf Lese- und Schreiboperationen gegeben. Hier kann man deutlich die geringere Größe des Level 1 Caches erkennen, da hier die Fehlerrate fast doppelt so hoch ist, wie im Cache des höchsten Levels.

Zuletzt wird in den Zeilen 13 bis 14 noch die Statistik für den Cache des höchsten Levels angegeben. Auch hier werden die Werte für die Gesamtzahl der Cachezugriffe und die Anzahl der Misses, sowie die sich daraus ergebende Fehlerrate angegeben. 

 Durch immer größer werdende Caches und ein immer besseres Management der Caches, sind die Fehlerraten des Beipielprogramms relativ gering, was auch an dem verhältnismäßig kleinen Speicherbereich liegt, der in dem Beispiel verwendet wurde.
 
 \begin{singlespace}
 \begin{scriptsize}
 \begin{lstlisting}
 ==5965== I   refs:      10,185,455
 ==5965== I1  misses:           780
 ==5965== LLi misses:           766
 ==5965== I1  miss rate:       0.00%
 ==5965== LLi miss rate:       0.00%
 ==5965== 
 ==5965== D   refs:       6,081,859  (3,965,231 rd   + 2,116,628 wr)
 ==5965== D1  misses:         8,370  (    1,930 rd   +     6,440 wr)
 ==5965== LLd misses:         7,907  (    1,495 rd   +     6,412 wr)
 ==5965== D1  miss rate:        0.1% (      0.0%     +       0.3%  )
 ==5965== LLd miss rate:        0.1% (      0.0%     +       0.3%  )
 ==5965== 
 ==5965== LL refs:            9,150  (    2,710 rd   +     6,440 wr)
 ==5965== LL misses:          8,673  (    2,261 rd   +     6,412 wr)
 ==5965== LL miss rate:         0.0% (      0.0%     +       0.3%  )
 \end{lstlisting}
 \end{scriptsize}
 \end{singlespace}
 
 Vergleicht man sie jedoch mit dem Beispiel in dem linear auf den Speicher zugegriffen wurde, so sind sie jedoch um Größenordnungen größer. Was in dem Beispiel noch als vernachlässigend gering erscheint, kann im allt"aglichen Gebrauch jedoch den Unterschied ausmachen, ob das Programm parallel zu anderen Programmen ausgeführt werden kann oder nicht.
 
 \begin{singlespace}
 \begin{scriptsize}
 \lstinputlisting{../Examples/cachegrind/cachegrind_example.c}
 \end{scriptsize}
 \end{singlespace}
 
 \section{Callgrind}
 Das Tool Callgrind ist eine Erweiterung des Tools Cachegrind. Neben der Funktionalit"at von Cachegrind erstellt es zus"atzlich einen Graphen, der zeigt, wie sich die Funktionen des Programms gegenseitig Aufrufen. Dies kann zum einen die Suche nach Misses im Befehls-Cache erleichtern. Zum anderen kann man so prüfen ob die gewünschte Architektur eingehalten wurde und eventuell ungewollte Abhängigkeiten erkennen und auflösen.
\newpage
\section{Helgrind}
Mit Helgrind wird ein Tool angeboten, welches den Programmierer beim programmieren von Multithread Programmen unterst"utzen soll. Es erkennt m"ogliche Deadlocks und Race-conditions. Ebenfalls wird auf eine falsche Verwendungen der POSIX pthreads API hingewiesen, sollte beispielsweise versucht werden, ein nicht gelocktes Mutex freizugeben.

Wenn Threads sich gegenseitig beeinflussen, kann es selten unter ung"unstigen Umst"anden zu einer gegenseitigen Beeinflussung kommen. Diese Fehler k"onnen so selten auftreten, dass sie mit etwas Pech beim Testen des Programms nicht gefunden werden. Mit Helgrind bietet die Valgrind-Suite ein Tool, dass solche Race-Conditions erkennt.

\subsection{Beispiel}
Im folgenden Beispielprogramm wird eine solche Race-Condition erschaffen. Das Programm selbst kann ohne Warnungen des Compilers gebaut werden und funktionierte bei einigen manuellen Test stets fehlerfrei.
\begin{singlespace}
\begin{scriptsize}
\lstinputlisting{../Examples/helgrind/dataRace.cpp}

\begin{lstlisting}
==2937== Possible data race during write of size 4 at 0xBEC690D4 by thread #3
==2937== Locks held: none
==2937==    at 0x8049060: Ship::loadContainers(int) (dataRace.cpp:20)
==2937==    by 0x804AA91: void std::_Mem_fn<void (Ship::*)(int)>::operator()<int, void>(Ship*, int&&) const (in /home/tobias/Qualit-tssicherung/Examples/helgrind/dataRace)
==2937==    by 0x804AA06: void std::_Bind_simple<std::_Mem_fn<void (Ship::*)(int)> (Ship*, int)>::_M_invoke<0u, 1u>(std::_Index_tuple<0u, 1u>) (functional:1732)
==2937==    by 0x804A908: std::_Bind_simple<std::_Mem_fn<void (Ship::*)(int)> (Ship*, int)>::operator()() (functional:1720)
==2937==    by 0x804A8BD: std::thread::_Impl<std::_Bind_simple<std::_Mem_fn<void (Ship::*)(int)> (Ship*, int)> >::_M_run() (thread:115)
==2937==    by 0x40F5D1D: ??? (in /usr/lib/i386-linux-gnu/libstdc++.so.6.0.19)
==2937==    by 0x402E661: ??? (in /usr/lib/valgrind/vgpreload_helgrind-x86-linux.so)
==2937==    by 0x402E661: ??? (in /usr/lib/valgrind/vgpreload_helgrind-x86-linux.so)
==2937==    by 0x415DF71: start_thread (pthread_create.c:312)
==2937==    by 0x426243D: clone (clone.S:129)
==2937== 
==2937== This conflicts with a previous write of size 4 by thread #2
==2937== Locks held: none
==2937==    at 0x8049060: Ship::loadContainers(int) (dataRace.cpp:20)
==2937==    by 0x804AA91: void std::_Mem_fn<void (Ship::*)(int)>::operator()<int, void>(Ship*, int&&) const (in /home/tobias/Qualit-tssicherung/Examples/helgrind/dataRace)
==2937==    by 0x804AA06: void std::_Bind_simple<std::_Mem_fn<void (Ship::*)(int)> (Ship*, int)>::_M_invoke<0u, 1u>(std::_Index_tuple<0u, 1u>) (functional:1732)
==2937==    by 0x804A908: std::_Bind_simple<std::_Mem_fn<void (Ship::*)(int)> (Ship*, int)>::operator()() (functional:1720)
==2937==    by 0x804A8BD: std::thread::_Impl<std::_Bind_simple<std::_Mem_fn<void (Ship::*)(int)> (Ship*, int)> >::_M_run() (thread:115)
==2937==    by 0x40F5D1D: ??? (in /usr/lib/i386-linux-gnu/libstdc++.so.6.0.19)
==2937==    by 0x402E661: ??? (in /usr/lib/valgrind/vgpreload_helgrind-x86-linux.so)
==2937==    by 0x402E661: ??? (in /usr/lib/valgrind/vgpreload_helgrind-x86-linux.so)
==2937==  Address 0xbec690d4 is on thread #1's stack
==2937==  in frame #6, created by loadShip() (dataRace.cpp:26)

\end{lstlisting}
\end{scriptsize}
\end{singlespace}

Das Tool Helgrind stellt allerdings Race-Conditions in nur einem Durchlauf sicher fest. Oben ist ein Auszug aus der Ausgabe zu sehen, in dem Helgrind die m"ogliche Race-Condition zwischen zwei der Threads angibt. F"ur die Race-Conditions mit den anderen Threads wurde jeweils eine "ahnliche Ausgabe erzeugt.
Die ausf"uhrlichen Callstacks helfen bei gro�en Programmen, den zusammenhang zwischen den Threads aufzul"osen und erm"oglichen es dem Nutzer die Race-Condition nachzuvollziehen.
\end{document}